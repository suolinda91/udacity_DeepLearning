\section{Building Generative Adversarial Networks}

\chapter{Introduction to Generative Adversarial Networks}

\section{Welcome to Generative Adversarial Networks}

\textbf{Quiz Question}

In simple terms, what is a Generative Adversarial Network (GAN)?
\begin{itemize}
    \item \textbf{A GAN is a deep learning model tasked with replicating a real distribution of data.}
    \item A GAN is a supervised learning model where data is classified to pre-determined labels.
\end{itemize}

With GANs we use computer-generated data with the goal of mimicking real data as closely as possible so that telling the difference between "real" and "fake" data becomes especially difficult.

\section{Course Outline}
\href{https://www.youtube.com/watch?v=Kv6-XVkTb14&t=4s}{Youtube} \newline
In this course, we will cover the following topics:

\begin{itemize}
    \item \textbf{Generative Adversarial Networks}
    \begin{itemize}
        \item Build a generator network
        \item Build a discriminator network
        \item Build GAN losses
        \item Train on the MNIST dataset
    \end{itemize}
    \item \textbf{Training a Deep Convolutional GANs}
    \begin{itemize}
        \item Build a DCGAN generator and discriminator
        \item Train a DCGAN model on the CIFAR10 dataset
        \item Implement GAN evaluation metrics
    \end{itemize}
    \item \textbf{Image to Image Translation}
    \begin{itemize}
        \item Implement CycleGAN dataloaders
        \item Implement CycleGAN generator and loss functions
        \item Train a CycleGAN
    \end{itemize}
    \item \textbf{Modern GANs: WGAN-GP, ProGAN and StyleGAN}
    \begin{itemize}
        \item Implement the Wasserstein Loss with gradient penalties
        \item Implement a ProGAN model
        \item Implement the components of a StyleGAN model
    \end{itemize}
\end{itemize}

\section{Prerequisites \& Tools}
\href{https://www.youtube.com/watch?v=hPKDrtwDj9g&t=1s}{Youtube} \newline

\subsection{Who Should Take This Course}

This advanced course is intended for students who have basic knowledge and experience in the following areas:

\begin{itemize}
    \item \textbf{Calculus, linear algebra, and probabilities}
    \item \textbf{Intermediate level in Python and PyTorch}
    \item \textbf{Building custom neural network architecture}
    \begin{itemize}
        \item Convolution and pooling layers
        \item Batch normalization
        \item Linear layers and different activation functions
    \end{itemize}
    \item \textbf{Training to convergence a neural network}
    \begin{itemize}
        \item PyTorch dataset and dataloaders
        \item Loss functions and optimizers
        \item Hyperparameter optimization
    \end{itemize}
\end{itemize}
If you are unfamiliar with these topics or need a refresher course, check out Udacity's free courses, \href{https://www.udacity.com/course/deep-learning-pytorch--ud188}{\textbf{Intro to Deep Learning with PyTorch}} and \href{https://www.udacity.com/course/linear-algebra-refresher-course--ud953}{\textbf{Linear Algebra Refresher}}.

\subsection{Tools \& Environment}
In this course, the work for all exercises and the course final project is contained in Jupyter Notebooks found in the workspaces provided in the classroom. \newline

We highly recommend completing coursework within the provided classroom workspaces. However, if you choose to work locally, the relevant files may be downloaded from the classroom workspace.

\section{Business Stakeholders}
\href{https://www.youtube.com/watch?v=eN_X5kgG9V4&t=19s}{Youtube} \newline

GANs are a fundamental component of computer-generated data. \newline

GANs can be used to:

\begin{itemize}
    \item \textbf{Generate interactive images from simple drawings}
    \item \textbf{Make synthetic data more realistic}
    \item \textbf{Create examples that can fool an existing algorithm}
\end{itemize}
Because of its broad range of existing and potential use cases, GANs can be applicable to virtually every industry, from retail to self-driving cars, from industrial to pharmaceuticals.


\section{Project Preview}
\href{https://www.youtube.com/watch?v=X8mFTCCYKB8&t=17s}{Youtube} \newline

For the final project in this course, you will:

\begin{itemize}
    \item \textbf{Build a custom generative adversarial network to generate new images of faces.}
    \item \textbf{Use a dataset of high-resolution images of "celebrity" faces.}
\end{itemize}
The project lesson contains a more detailed overview of the project, as well as, detailed instructions.