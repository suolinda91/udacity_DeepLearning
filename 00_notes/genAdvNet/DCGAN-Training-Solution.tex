\section{Deep Convolutional GANs}

In this notebook, you'll build a GAN using convolutional layers in the
generator and discriminator. This is called a Deep Convolutional GAN, or
DCGAN for short. The DCGAN architecture was first explored in 2016 and
has seen impressive results in generating new images; you can read the
\href{https://arxiv.org/pdf/1511.06434.pdf}{original paper, here}.

You'll be training DCGAN on the
\href{https://www.cs.toronto.edu/~kriz/cifar.html}{CIFAR10} dataset.
These are color images of different classes, such as airplanes, dogs or
trucks. This dataset is much more complex and diverse than the MNIST
dataset and justifies the use of the DCGAN architecture.

So, our goal is to create a DCGAN that can generate new,
realistic-looking images. We'll go through the following steps to do
this: 
\begin{itemize}
    \item \textbf{Load in and pre-process the CIFAR10 dataset}
    \item Define discriminator and generator networks
    \item \textbf{Train these adversarial networks}
    \item ƒ\textbf{Visualize the loss over time and some sample, generated images}
\end{itemize}

In this notebook, we will focus on defining the networks.

\paragraph{Deeper Convolutional Networks}

Since this dataset is more complex than our MNIST data, we'll need a
deeper network to accurately identify patterns in these images and be
able to generate new ones. Specifically, we'll use a series of
convolutional or transpose convolutional layers in the discriminator and
generator. It's also necessary to use batch normalization to get these
convolutional networks to train.

Besides these changes in network structure, training the discriminator
and generator networks should be the same as before. That is, the
discriminator will alternate training on real and fake (generated)
images, and the generator will aim to trick the discriminator into
thinking that its generated images are real!

\begin{lstlisting}[language=Python]
# # run this cell once to install the dependency. 
# You will have to restart the kernel once the package is installed.
# !pip install ipywidgets
\end{lstlisting}

\begin{lstlisting}[language=Python]
# import libraries
import matplotlib.pyplot as plt
import numpy as np
import pickle as pkl

%matplotlib inline
\end{lstlisting}

\subsection{Getting the data}

Here you can download the CIFAR10 dataset. It's a dataset built-in to
the PyTorch datasets library. We can load in training data, transform it
into Tensor datatypes, then create dataloaders to batch our data into a
desired size.

\begin{lstlisting}[language=Python]
import torch
from torchvision import datasets
from torchvision import transforms

# Tensor transform
transform = transforms.ToTensor()

# CIFAR training datasets
cifar_train = datasets.CIFAR10(root='data/', train=True, download=True, transform=transform)

batch_size = 128
num_workers = 4

# build DataLoaders for CIFAR10 dataset
train_loader = torch.utils.data.DataLoader(dataset=cifar_train,
                                          batch_size=batch_size,
                                          shuffle=True,
                                          num_workers=num_workers)
\end{lstlisting}

\subsubsection{Visualize the Data}

Here I'm showing a small sample of the images. Each of these is 32x32
with 3 color channels (RGB). These are the real, training images that
we'll pass to the discriminator. Notice that each image has \emph{one}
associated, numerical label.

\begin{lstlisting}[language=Python]
# obtain one batch of training images
dataiter = iter(train_loader)
images, labels = dataiter.next()

# plot the images in the batch, along with the corresponding labels
fig = plt.figure(figsize=(25, 4))
plot_size=20
for idx in np.arange(plot_size):
    ax = fig.add_subplot(2, plot_size/2, idx+1, xticks=[], yticks=[])
    ax.imshow(np.transpose(images[idx], (1, 2, 0)))
    # print out the correct label for each image
    # .item() gets the value contained in a Tensor
    ax.set_title(str(labels[idx].item()))
\end{lstlisting}

\subsubsection{Pre-processing: scaling from -1 to 1}

We need to do a bit of pre-processing; we know that the output of our
\lstinline{tanh} activated generator will contain pixel
values in a range from -1 to 1, and so, we need to rescale our training
images to a range of -1 to 1. (Right now, they are in a range from 0-1.)

\begin{lstlisting}[language=Python]
# current range
img = images[0]

print('Min: ', img.min())
print('Max: ', img.max())
\end{lstlisting}

\begin{lstlisting}[language=Python]
# helper scale function
def scale(x, feature_range=(-1, 1)):
    ''' Scale takes in an image x and returns that image, scaled
       with a feature_range of pixel values from -1 to 1. 
       This function assumes that the input x is already scaled from 0-1.'''
    # assume x is scaled to (0, 1)
    # scale to feature_range and return scaled x
    min_val, max_val = feature_range
    x = x * (max_val - min_val) + min_val
    return x
\end{lstlisting}

\begin{lstlisting}[language=Python]
# scaled range
scaled_img = scale(img)

print('Scaled min: ', scaled_img.min())
print('Scaled max: ', scaled_img.max())
\end{lstlisting}

\section{Define the Model}

A GAN is comprised of two adversarial networks, a discriminator and a
generator. Let's use the models we created in the previous exercise.

\subsection{Discriminator}

Here you'll build the discriminator. This is a convolutional classifier
like you've built before, only without any maxpooling layers. 
\begin{itemize}
    \item The inputs to the discriminator are 32x32x3 tensor images
    \item You'll want a few convolutional, hidden layers
    \item Then a fully connected layer for the output; as before, we want a sigmoid output, but we'll add that in the loss function, \href{https://pytorch.org/docs/stable/nn.html\#bcewithlogitsloss}{BCEWithLogitsLoss}, later
\end{itemize}

\begin{lstlisting}[language=Python]
import torch.nn as nn
\end{lstlisting}

\begin{lstlisting}[language=Python]
class ConvBlock(nn.Module):
    """
    A convolutional block is made of 3 layers: Conv -> BatchNorm -> Activation.
    args:
    - in_channels: number of channels in the input to the conv layer
    - out_channels: number of filters in the conv layer
    - kernel_size: filter dimension of the conv layer
    - batch_norm: whether to use batch norm or not
    """
    def __init__(self, in_channels: int, out_channels: int, kernel_size: int, batch_norm: bool = True):
        super(ConvBlock, self).__init__()
        
        self.conv = nn.Conv2d(in_channels, out_channels, kernel_size, stride=2, padding=1, bias=False)
        self.batch_norm = batch_norm
        if self.batch_norm:
            self.bn = nn.BatchNorm2d(out_channels)
        self.activation = nn.LeakyReLU(0.2)
        
    def forward(self, x: torch.Tensor) -> torch.Tensor:
        x = self.conv(x)
        if self.batch_norm:
            x = self.bn(x)
        x = self.activation(x)
        return x
\end{lstlisting}

\begin{lstlisting}[language=Python]
class Discriminator(nn.Module):
    """
    The discriminator model adapted from the DCGAN paper. It should only contains a few layers.
    args:
    - conv_dim: control the number of filters
    """
    def __init__(self, conv_dim: int):
        super(Discriminator, self).__init__()

        # complete init function
        self.conv_dim = conv_dim

        # 32x32 input
        self.conv1 = ConvBlock(3, conv_dim, 4, batch_norm=False) # first layer, no batch_norm
        # 16x16 out
        self.conv2 = ConvBlock(conv_dim, conv_dim*2, 4)
        # 8x8 out
        self.conv3 = ConvBlock(conv_dim*2, conv_dim*4, 4)
        # 4x4 out
        
        self.flatten = nn.Flatten()
        # final, fully-connected layer
        self.fc = nn.Linear(conv_dim*4*4*4, 1)

    def forward(self, x):
        # all hidden layers + leaky relu activation
        x = self.conv1(x)
        x = self.conv2(x)
        x = self.conv3(x)
        # flatten
        x = self.flatten(x)
        # final output layer
        x = self.fc(x)        
        return x
\end{lstlisting}

\subsection{Generator}

Next, you'll build the generator network. The input will be our noise
vector \lstinline{z}, as before. And, the output will be a
\(tanh\) output, but this time with size 32x32 which is the size of our
CIFAR10 images.

\begin{lstlisting}[language=Python]
class DeconvBlock(nn.Module):
    """
    A "de-convolutional" block is made of 3 layers: ConvTranspose -> BatchNorm -> Activation.
    args:
    - in_channels: number of channels in the input to the conv layer
    - out_channels: number of filters in the conv layer
    - kernel_size: filter dimension of the conv layer
    - stride: stride of the conv layer
    - padding: padding of the conv layer
    - batch_norm: whether to use batch norm or not
    """
    def __init__(self, 
                 in_channels: int, 
                 out_channels: int, 
                 kernel_size: int, 
                 stride: int,
                 padding: int,
                 batch_norm: bool = True):
        super(DeconvBlock, self).__init__()
        self.deconv = nn.ConvTranspose2d(in_channels, out_channels, kernel_size, stride, padding, bias=False)
        self.batch_norm = batch_norm
        if self.batch_norm:
            self.bn = nn.BatchNorm2d(out_channels)
        self.activation = nn.ReLU()
        
    def forward(self, x: torch.Tensor) -> torch.Tensor:
        x = self.deconv(x)
        if self.batch_norm:
            x = self.bn(x)
        x = self.activation(x)
        return x
\end{lstlisting}

\begin{lstlisting}[language=Python]
class Generator(nn.Module):
    """
    The generator model adapted from DCGAN
    args:
    - latent_dim: dimension of the latent vector
    - conv_dim: control the number of filters in the convtranspose layers
    """
    def __init__(self, latent_dim: int, conv_dim: int = 32):
        super(Generator, self).__init__()
        # transpose conv layers
        self.deconv1 = DeconvBlock(latent_dim, conv_dim*4, 4, 1, 0)
        self.deconv2 = DeconvBlock(conv_dim*4, conv_dim*2, 4, 2, 1)
        self.deconv3 = DeconvBlock(conv_dim*2, conv_dim, 4, 2, 1)
        self.deconv4 = nn.ConvTranspose2d(conv_dim, 3, 4, stride=2, padding=1)
        self.last_activation = nn.Tanh()
        
    def forward(self, x):
        x = self.deconv1(x)
        x = self.deconv2(x)
        x = self.deconv3(x)
        x = self.deconv4(x)
        x = self.last_activation(x)
        return x
    
\end{lstlisting}

\subsection{Build complete network}

Define your models' hyperparameters and instantiate the discriminator
and generator from the classes defined above. Make sure you've passed in
the correct input arguments.

\begin{lstlisting}[language=Python]
# define hyperparams
conv_dim = 32
z_size = 100

# define discriminator and generator
D = Discriminator(conv_dim)
G = Generator(latent_dim=z_size, conv_dim=conv_dim)
\end{lstlisting}

\subsubsection{Training on GPU}

Check if you can train on GPU. If you can, set this as a variable and
move your models to GPU. \textgreater{} Later, we'll also move any
inputs our models and loss functions see (real\_images, z, and ground
truth labels) to GPU as well.

\begin{lstlisting}[language=Python]
train_on_gpu = torch.cuda.is_available()

if train_on_gpu:
    # move models to GPU
    G.cuda()
    D.cuda()
    print('GPU available for training. Models moved to GPU')
else:
    print('Training on CPU.')
    
\end{lstlisting}

\subsection{Discriminator and Generator Losses}

Now we need to calculate the losses. And this will be exactly the same
as before.

\subsubsection{Discriminator Losses}

\begin{quote}
\begin{itemize}
\item For the discriminator, the total loss is the sum of the losses for real and fake images,  \lstinline{d_loss = d_real_loss + d_fake_loss}.
\item Remember that we want the discriminator to output 1 for real images and 0 for fake images, so we need to set up the losses to reflect that.
\end{itemize}
\end{quote}

The losses will by binary cross entropy loss with logits, which we can
get with
\href{https://pytorch.org/docs/stable/nn.html\#bcewithlogitsloss}{BCEWithLogitsLoss}.
This combines a \lstinline{sigmoid} activation function
\textbf{and} binary cross entropy loss in one function.

For the real images, we want
\lstinline{D(real_images) = 1}. That is, we want the
discriminator to classify the real images with a label = 1, indicating
that these are real. The discriminator loss for the fake data is
similar. We want \lstinline{D(fake_images) = 0}, where
the fake images are the \emph{generator output},
\lstinline{fake_images = G(z)}.

\subsubsection{Generator Loss}

The generator loss will look similar only with flipped labels. The
generator's goal is to get
\lstinline{D(fake_images) = 1}. In this case, the labels
are \textbf{flipped} to represent that the generator is trying to fool
the discriminator into thinking that the images it generates (fakes) are
real!

\begin{lstlisting}[language=Python]
def real_loss(D_out, smooth=False):
    batch_size = D_out.size(0)
    # label smoothing
    if smooth:
        # smooth, real labels = 0.9
        labels = torch.ones(batch_size)*0.9
    else:
        labels = torch.ones(batch_size) # real labels = 1
    # move labels to GPU if available     
    if train_on_gpu:
        labels = labels.cuda()
    # binary cross entropy with logits loss
    criterion = nn.BCEWithLogitsLoss()
    # calculate loss
    loss = criterion(D_out.squeeze(), labels)
    return loss

def fake_loss(D_out):
    batch_size = D_out.size(0)
    labels = torch.zeros(batch_size) # fake labels = 0
    if train_on_gpu:
        labels = labels.cuda()
    criterion = nn.BCEWithLogitsLoss()
    # calculate loss
    loss = criterion(D_out.squeeze(), labels)
    return loss
\end{lstlisting}

\subsection{Optimizers}

Not much new here, but notice how I am using a small learning rate and
custom parameters for the Adam optimizers, This is based on some
research into DCGAN model convergence.

\subsubsection{Hyperparameters}

GANs are very sensitive to hyperparameters. A lot of experimentation
goes into finding the best hyperparameters such that the generator and
discriminator don't overpower each other. Try out your own
hyperparameters or read \href{https://arxiv.org/pdf/1511.06434.pdf}{the
DCGAN paper} to see what worked for them.

\begin{lstlisting}[language=Python]
import torch.optim as optim

# params
lr = 0.0002
beta1=0.5
beta2=0.999 # default value

# Create optimizers for the discriminator and generator
d_optimizer = optim.Adam(D.parameters(), lr, [beta1, beta2])
g_optimizer = optim.Adam(G.parameters(), lr, [beta1, beta2])
\end{lstlisting}

\subsection{Training}

Training will involve alternating between training the discriminator and
the generator. We'll use our functions
\lstinline{real_loss} and
\lstinline{fake_loss} to help us calculate the
discriminator losses in all of the following cases.

\subsubsection{Discriminator training}

\begin{enumerate}
\item Compute the discriminator loss on real, training images\\
\item Generate fake images
\item Compute the discriminator loss on fake, generated images\\
\item Add up real and fake loss
\item Perform backpropagation + an optimization step to update the discriminator's weights
\end{enumerate}

\subsubsection{Generator training}

\begin{enumerate}
\item Generate fake images
\item Compute the discriminator loss on fake images, using \textbf{flipped} labels!
\item Perform backpropagation + an optimization step to update the generator's weights
\end{enumerate}

\paragraph{Saving Samples}

As we train, we'll also print out some loss statistics and save some
generated ``fake'' samples.

\textbf{Evaluation mode}

Notice that, when we call our generator to create the samples to
display, we set our model to evaluation mode:
\lstinline{G.eval()}. That's so the batch normalization
layers will use the population statistics rather than the batch
statistics (as they do during training), \emph{and} so dropout layers
will operate in eval() mode; not turning off any nodes for generating
samples.

\begin{lstlisting}[language=Python]
import pickle as pkl
from datetime import datetime
\end{lstlisting}

\begin{lstlisting}[language=Python]
# helper function for viewing a list of passed in sample images
def view_samples(epoch, samples):
    fig, axes = plt.subplots(figsize=(14,4), nrows=2, ncols=8, sharey=True, sharex=True)
    for ax, img in zip(axes.flatten(), samples[epoch]):
        img = img.detach().cpu().numpy()
        img = np.transpose(img, (1, 2, 0))
        img = ((img +1)*255 / (2)).astype(np.uint8) # rescale to pixel range (0-255)
        ax.xaxis.set_visible(False)
        ax.yaxis.set_visible(False)
        im = ax.imshow(img.reshape((32,32,3)))
    plt.show()
\end{lstlisting}

\begin{lstlisting}[language=Python]
# training hyperparams
num_epochs = 10

# keep track of loss and generated, "fake" samples
samples = []
losses = []

print_every = 100

# Get some fixed data for sampling. These are images that are held
# constant throughout training, and allow us to inspect the model's performance
sample_size=16
fixed_z = np.random.uniform(-1, 1, size=(sample_size, z_size, 1, 1))
fixed_z = torch.from_numpy(fixed_z).float()

# train the network
for epoch in range(num_epochs):
    
    for batch_i, (real_images, _) in enumerate(train_loader):
                
        batch_size = real_images.size(0)
        
        # important rescaling step
        real_images = scale(real_images)
        
        # ============================================
        #            TRAIN THE DISCRIMINATOR
        # ============================================
        
        d_optimizer.zero_grad()
        
        # 1. Train with real images

        # Compute the discriminator losses on real images 
        if train_on_gpu:
            real_images = real_images.cuda()
        
        D_real = D(real_images)
        d_real_loss = real_loss(D_real)
        
        # 2. Train with fake images
        
        # Generate fake images
        z = np.random.uniform(-1, 1, size=(batch_size, z_size, 1, 1))
        z = torch.from_numpy(z).float()
        # move x to GPU, if available
        if train_on_gpu:
            z = z.cuda()
        fake_images = G(z)
        
        # Compute the discriminator losses on fake images            
        D_fake = D(fake_images.detach())
        d_fake_loss = fake_loss(D_fake)
        
        # add up loss and perform backprop
        d_loss = d_real_loss + d_fake_loss
        d_loss.backward()
        d_optimizer.step()
        
        
        # =========================================
        #            TRAIN THE GENERATOR
        # =========================================
        g_optimizer.zero_grad()
        
        # 1. Train with fake images and flipped labels
        
        # Generate fake images
        z = np.random.uniform(-1, 1, size=(batch_size, z_size, 1, 1))
        z = torch.from_numpy(z).float()
        if train_on_gpu:
            z = z.cuda()
        fake_images = G(z)
        
        # Compute the discriminator losses on fake images 
        # using flipped labels!
        D_fake = D(fake_images)
        g_loss = real_loss(D_fake) # use real loss to flip labels
        
        # perform backprop
        g_loss.backward()
        g_optimizer.step()

        # Print some loss stats
        if batch_i % print_every == 0:
            # append discriminator loss and generator loss
            losses.append((d_loss.item(), g_loss.item()))
            # print discriminator and generator loss
            time = str(datetime.now()).split('.')[0]
            print(f'{time} | Epoch [{epoch+1}/{num_epochs}] | Batch {batch_i}/{len(train_loader)} | d_loss: {d_loss.item():.4f} | g_loss: {g_loss.item():.4f}')
    
    ## AFTER EACH EPOCH##    
    # generate and save sample, fake images
    G.eval() # for generating samples
    if train_on_gpu:
        fixed_z = fixed_z.cuda()
    samples_z = G(fixed_z)
    samples.append(samples_z)
    view_samples(-1, samples)
    G.train() # back to training mode


# Save training generator samples
with open('train_samples.pkl', 'wb') as f:
    pkl.dump(samples, f)
\end{lstlisting}

\subsection{Training loss}

Here we'll plot the training losses for the generator and discriminator,
recorded after each epoch.

\begin{lstlisting}[language=Python]
fig, ax = plt.subplots()
losses = np.array(losses)
plt.plot(losses.T[0], label='Discriminator', alpha=0.5)
plt.plot(losses.T[1], label='Generator', alpha=0.5)
plt.title("Training Losses")
plt.legend()
\end{lstlisting}
