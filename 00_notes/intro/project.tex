
\section{Project Overview}

\subsection{MNIST Digit Classifier Project Introduction}
As a machine learning engineer, you’ve gained familiarity with deep learning, a powerful tool for a number of tasks – not the least of which is computer vision. As part of a new product, you’ve been asked to prototype a system for optical character recognition (OCR) on handwritten characters. Since the team is still collecting samples of data, you’ve been tasked with providing a proof of concept on the MNIST database of handwritten digits, a task with very similar input and output.

In this project, you will be given a Jupyter Notebook to do all of your coding and written explanations. You will preprocess a dataset for handwritten digit recognition, build a neural network, then train and tune that neural network using your data.

Before you get started, please see the \href{https://learn.udacity.com/rubric/4816}{\textbf{rubric}}, and be sure to see the Jupyter Notebook on the Environments to get started.

\section{Instructions}

\subsection{Project Instructions}

\subsection{Submission Instructions}

\textit{Please save your trained model in the workspace or upload a single file for review.}

\subsection{Instructions Summary}

\subsection{Step 1}

\begin{itemize}
    \item Load the dataset from \verb|torchvision.datasets|
    \item Use transforms or other PyTorch methods to convert the data to tensors, normalize, and flatten the data.
    \item Create a \verb|DataLoader| for your dataset
\end{itemize}

\subsection{Step 2}

\begin{itemize}
    \item Visualize the dataset using the provided function and either:

\begin{itemize}
        \item Your training data loader and inverting any normalization and flattening
        \item A second \verb|DataLoader| without any normalization or flattening
\end{itemize}

    \item Explore the size and shape of the data to get a sense of what your inputs look like naturally and after transformation. Provide a brief justification of any necessary preprocessing steps or why no preprocessing is needed.
\end{itemize}

\subsection{Step 3}

\begin{itemize}
    \item Using PyTorch, build a neural network to predict the class of each given input image
    \item Create an optimizer to update your network’s weights
    \item Use the training \verb|DataLoader| to train your neural network
\end{itemize}

\subsection{Step 4}

\begin{itemize}
    \item Evaluate your neural network’s accuracy on the test set.
    \item Tune your model hyperparameters and network architecture to improve your test set accuracy, achieving at least 90\% accuracy on the test set.
\end{itemize}

\subsection{Step 5}

Use \verb|torch.save| to save your trained model.
